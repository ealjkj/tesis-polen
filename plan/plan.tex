\chapter{Planeación}
    Éste no es un capítulo de la tesis. Sólo está en el mismo documento para tener mis ideas apuntadas.
    He decidido que los capítulos del "libro" deben cubrir los siguientes temas. El órden y la distribución de los capítulos están sujetos a cambios.
    \begin{enumerate}
        \item \textbf{Estado del arte:} Aquí irán los estudios que preceden al mío. Incluso pienso escribir resultados previos para clasificación de pollen. 
        \item \textbf{Preeliminares:} Con el objetivo de que este trabajo sea autocontenido, pretendo escribir cualquier tema que se use en algún capítulo. En ésta sección sólo se mencionarán algunos temas, sin necesidad de indagar a fondo en ellos. Ejmplos de esto podrían ser \textsl{Fully Connected Network} o algunos resultados de cálculo variacional o Ecuaciones diferenciales.
        \begin{enumerate}
            \item Series de Taylor
        \end{enumerate}
        \item \textbf{Redes neuronales convolucionales:} Una explicación de cómo funcionan. De cómo están construídas y la notación matemática destras de éstas. Me gustaría hacer énfasis en que una convolución $X*k$ puede ser visto como una multiplicación de matrices $Ax$.
        \item  \textbf{Métodos de ecuaciones diferenciales:} El objetivo en este capítulo es describir los métodos más comunes de ecuaciones diferenciales, cómo lo son Euler, El método del trapecio o Runge Kutta. Quizá incluso se podría tomar en cuenta los estudios de estabilidad de los métodos.
        \item  \textbf{Teoría de control óptimo.} Una breve explicación del control óptimo. El principio del máximo de Pontryagin, las ecuaciones de Hammilton Bellman. Los métodos de resolución comunes, como el shooting method. Y sobre su relación con el entrenamiento de una red neuronal.
        \item \textbf{ResNet:} Tanto la explicación de una, como su interpretación como discretización de una ODE.
        \item \textbf{Estudio de la estabilidad de redes neuronales:} Éste capítulo está dedicado para los resultados teóricos que obtengamos como resultado de crear una arquitectura nueva, o modificar la ResNet.
        \item  \textbf{Experimentación;} Los resultados de nuestra arquitectura comparada con otras arquitecturas en el estado del arte.
        \item \textbf{Conclusiones:} Un breve resumen de los resultados obtenidos.
    \end{enumerate}
    No necesariamente son 9 capítulos. Algunos podrían entrar dentro de los preeliminares y algunos puntos podrían formar un sólo capítulo en conjunto.